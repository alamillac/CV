\documentclass[11pt,a4paper]{moderncv}

% moderncv themes
\moderncvtheme[blue]{classic}                 % optional argument are 'blue' (default), 'orange', 'red', 'green', 'grey' and 'roman' (for roman fonts, instead of sans serif fonts)
%\moderncvtheme[green]{classic}                % idem

% character encoding
\usepackage[utf8]{inputenc}                   % replace by the encoding you are using

% adjust the page margins
\usepackage[scale=0.8]{geometry}
%\setlength{\hintscolumnwidth}{3cm}                     % if you want to change the width of the column with the dates
%\AtBeginDocument{\setlength{\maketitlenamewidth}{6cm}}  % only for the classic theme, if you want to change the width of your name placeholder (to leave more space for your address details
%\AtBeginDocument{\recomputelengths}                     % required when changes are made to page layout lengths

% personal data
\firstname{Andrés}
\familyname{Lamilla}
\title{Ingeniero Electrónico}               % optional, remove the line if not wanted
\address{Calle Corcega 573}{08025 Barcelona (España)}    % optional, remove the line if not wanted
%\mobile{+34 644325458}                    % optional, remove the line if not wanted
%\phone{+571 6958419}                      % optional, remove the line if not wanted
%\fax{fax (optional)}                          % optional, remove the line if not wanted
\email{andres@lamillac.com}                      % optional, remove the line if not wanted
%\homepage{homepage (optional)}                % optional, remove the line if not wanted
%\extrainfo{additional information (optional)} % optional, remove the line if not wanted
\photo[64pt]{foto}                         % '64pt' is the height the picture must be resized to and 'picture' is the name of the picture file; optional, remove the line if not wanted
\quote{Entusiasta de la robótica y la programación. Amante de los retos y desafíos. Interesado en la investigación y desarrollo de soluciones informáticas. Con grandes deseos por aplicar y aprender nuevos temas cada día. Con experiencia en manejo de servidores linux, programación en diferentes lenguajes y aplicacion de algoritmos de aprendizaje automático.}                 % optional, remove the line if not wanted

% to show numerical labels in the bibliography; only useful if you make citations in your resume
%\makeatletter
%\renewcommand*{\bibliographyitemlabel}{\@biblabel{\arabic{enumiv}}}
%\makeatother

% bibliography with mutiple entries
%\usepackage{multibib}
%\newcites{book,misc}{{Books},{Others}}

%\nopagenumbers{}                             % uncomment to suppress automatic page numbering for CVs longer than one page
%----------------------------------------------------------------------------------
%            content
%----------------------------------------------------------------------------------
\begin{document}
\maketitle

%\section{Perfil Profesional}

\section{Formación académica}
\cventry{2012--}{Master in Artificial Intelligence}{Universidad Politecnica de Catalunya}{Barcelona}{}{}
\cventry{2006--2010}{Ingeniería Electrónica}{Escuela Colombiana de Ingeniería}{Bogotá}{}{}  % arguments 3 to 6 can be left empty
%\cventry{year--year}{Degree}{Institution}{City}{\textit{Grade}}{Description}

\section{Formación complementaria}
\cventry{2012}{Curso de seguridad informatica}{Jedi junior empresa}{Barcelona}{}{}
\cventry{2012}{Building a Search Engine}{Curso gratuito ofrecido por Udacity}{Online}{}{http://www.udacity.com/}
\cventry{2011}{Introduction to Artificial Intelligence}{Curso gratuito ofrecido por Stanford}{Online}{}{https://www.ai-class.com/}
\cventry{2011}{Machine Learning}{Curso gratuito ofrecido por Stanford}{Online}{}{http://www.ml-class.org/course/auth/welcome}
%\cventry{2011}{Introduction to Databases}{Curso gratuito ofrecido por Stanford}{Online}{}{http://www.db-class.org/course/auth/welcome}
\cventry{2011}{Curso de inglés}{EF International Language Centers}{Seattle}{}{Realizado entre Abril y Septiembre}
\cventry{2010}{Simposio STSIVA}{Escuela Colombiana de Ingeniería}{Bogotá}{}{Simposio de tratamiento de señales, imágenes y visión artificial}
\cventry{2010}{Minicurso Visión infrarroja}{Escuela Colombiana de Ingeniería}{Bogotá}{}{Visión infrarroja: Teoría y aplicaciones, durante el simposio STSIVA}
\cventry{2009}{Curso de microcontroladores Freescale}{Escuela Colombiana de Ingeniería}{Bogotá}{}{Programación de microcontroladores Freescale organizado por la rama estudiantil de la IEEE}

\section{Experiencia}
\subsection{Profesional}
\cventry{2014-}{Programador web}{Visual Engineering}{Barcelona}{}{Durante mi estancia en Visual estuve encargado de implementar diferentes soluciones web y mobile, dentro de las cuales están: \newline{}Desarrollar una red social de familias usando el framework de python Django.\newline{}Corregir bugs e incidencias en la app mobile de Zara para Blackberry.\newline{}Incrementar la cobertura de tests para la app web mobile de Zara.\newline{}Desarrollar una single-page application de venta de camisas para Massimo Dutti.\newline{}Adición de algunas funcionalidades a app web para gas natural}
\cventry{2014-}{Programador web, Sys-admin}{Linkgua Semantic}{Barcelona}{}{Estuve en el equipo de desarrollo de esta aplicación web de lectura con análisis semantico. Me encargué de crear toda la estructura para ponerla en producción y gestionar los deploys. Ahora me encargo de mantener su correcto funcionamiento.}
\cventry{2012--2014}{Operaciones}{Spamina}{Barcelona}{}{Realización de scripts para automatización de tareas.\newline{}Monitorización y control de servidores.\newline{}Manejo de incidencias en los sistemas.}
\subsection{Otros}
\cventry{2011}{Especialista en sistemas de vigilancia}{Activar}{Bogota}{}{Instalación y mantenimiento de sistema de vigilancia en servidores con Zoneminder.}

\section{Idiomas}
\cvlanguage{Español}{Nativo}{}
\cvlanguage{Inglés}{Nivel medio}{}

\section{Premios}
\cventry{2009}{Robótica}{V Olimpiada Nacional de Robot Ecibot, Escuela Colombiana de Ingeniería}{Bogotá}{}{Primer puesto en la categoría seguidor de línea básico}

\closesection{}
%\pagebreak{}

\section{Conocimientos}
\cvcomputer{SO}{Gnu/Linux (Debian, Ubuntu, Archlinux, Mandriva, RedHat, Fedora, CentOS), FreeBSD, Windows}{Programación Web}{Django, Flask, PhalconPHP, Node, Backbonejs, Emberjs, jQuery, Google app engine}
\cvcomputer{Programación}{Javascript, C/C++, Python, PHP, Ruby, Matlab/Octave, Bash/Shell, Zsh/Shell, sed, \LaTeX, Java}{Bases de datos}{MySQL, MongoDB}
\cvcomputer{Diseño Web}{HTML5, CSS3, SASS}{Librerías IA}{OpenCV, FANN, scikit-learn}
\cvcomputer{Herramientas Linux}{Nagios, Ansible, Docker, KVM, Virt-manager, BackupPC, Dokuwiki, Wikimedia, Logstash/Kibana}{Utilidades Web}{RequireJS, Grunt, Bower, Mocha, Jasmine}

\section{Intereses}
\cvlistitem{Fútbol}
\cvlistitem{Hiking}
\cvlistitem{Ciclísmo}
\cvlistitem{Running}

\renewcommand{\listitemsymbol}{-} % change the symbol for lists

\section{Proyectos}
\cvlistitem{Aplicación web: Red internacional de familias https://www.ifamilynetworks.com}
\cvlistitem{Aplicación web: Lector de libros con analisis semantico https://www.linkgua-books.com}
\cvlistitem{Aplicación web: Lector de noticias con analisis semantico https://www.linkgua-semantic.com}
\cvlistitem{Puesta en funcionamiento y mantenimiento de servidores Linux para sistemas de vigilancia, realizado para Activar SA}
\cvlistitem{Manejo de dispositivos remotos mediante comandos gestuales, realizado para la materia de Vision Artificial en la Escuela Colombiana de Ingeniería}
\cvlistitem{Robot seguidor de linea y de resolución de laberintos, realizado para participar en la V Olimpiada Nacional de Robot Ecibot}
\cvlistitem{Collar de perros con localizador por GPS, Proyecto realizado en la Universidad}
\cvlistitem{Página web del juego tres en línea 3D con adversario del sistema para probar el algoritmo min max}
\cvlistitem{Reconocimiento de figuras geométricas y números de placas mediante cámara web. Proyecto realizado para reforzar el conocimiento de visión artificial}


%\section{Extra 1}
%\cvlistitem{Item 1}
%\cvlistitem{Item 2}
%\cvlistitem[+]{Item 3}            % optional other symbol

%\renewcommand{\listitemsymbol}{-} % change the symbol for lists

%\section{Extra 2}
%\cvlistdoubleitem{Item 1}{Item 4}
%\cvlistdoubleitem{Item 2}{Item 5 \cite{book1}}
%\cvlistdoubleitem{Item 3}{}

% Publications from a BibTeX file without multibib\renewcommand*{\bibliographyitemlabel}{\@biblabel{\arabic{enumiv}}}% for BibTeX numerical labels
%\nocite{*}
%\bibliographystyle{plain}
%\bibliography{publications}       % 'publications' is the name of a BibTeX file

% Publications from a BibTeX file using the multibib package
%\section{Publications}
%\nocitebook{book1,book2}
%\bibliographystylebook{plain}
%\bibliographybook{publications}   % 'publications' is the name of a BibTeX file
%\nocitemisc{misc1,misc2,misc3}
%\bibliographystylemisc{plain}
%\bibliographymisc{publications}   % 'publications' is the name of a BibTeX file

\end{document}


% end of file
